\documentclass[a4paper,12 pt]{report}
\usepackage[T1]{fontenc}
\usepackage[utf8]{inputenc}
\usepackage{lmodern}
\usepackage{listings}
\usepackage{graphicx}
\usepackage{float}
\usepackage{subcaption}
\usepackage{wrapfig}
\usepackage{fancyhdr}
\usepackage{amsthm}
\usepackage{amsmath}
\usepackage{amsfonts} 
\usepackage{pgfplots}
%todos
\setlength {\marginparwidth }{2cm}
\usepackage{todonotes}
\newcommand{\TODO}[2][]
{\todo[size=\scriptsize, color=red, #1]{#2}}

\pgfplotsset{compat=1.18}
% forza le footnote a stare il più in basso possibile
\usepackage[bottom]{footmisc}


%% STILE LISTINGS

\usepackage{xcolor}

\definecolor{codegreen}{rgb}{0,0.6,0}
\definecolor{codegray}{rgb}{0.5,0.5,0.5}
\definecolor{codepurple}{rgb}{0.58,0,0.82}
\definecolor{backcolour}{rgb}{0.95,0.95,0.92}

\lstdefinestyle{mystyle}{
    backgroundcolor=\color{backcolour},   
    commentstyle=\color{codegreen},
    keywordstyle=\color{magenta},
    numberstyle=\tiny\color{codegray},
    stringstyle=\color{codepurple},
    basicstyle=\ttfamily\footnotesize,
    breakatwhitespace=false,         
    breaklines=true,                 
    captionpos=b,                    
    keepspaces=true,                 
    numbers=left,                    
    numbersep=5pt,                  
    showspaces=false,                
    showstringspaces=false,
    showtabs=false,                  
    tabsize=2
}

\lstset{style=mystyle}

%% -----

% mostra le subsubsection nell'indice
\setcounter{tocdepth}{3}
\setcounter{secnumdepth}{3}

% Resetta la numerazione dei chapter quando
% una nuova part viene creata
\makeatletter
\@addtoreset{chapter}{part}
\makeatother

% Rimuove l'indentazione quando si crea un nuovo paragrafo
\setlength{\parindent}{0pt}

% footer
\pagestyle{fancyplain}
% rimuove la riga nell'header
\fancyhf{} % sets both header and footer to nothing
\renewcommand{\headrulewidth}{0pt}
\fancyfoot[L]{\href{https://github.com/Typing-Monkeys/AppuntiUniversita}{Typing Monkeys}}
\fancyfoot[C]{\emoji{gorilla}}
\fancyfoot[R]{\thepage}

% configurazione emoji
\usepackage{fontspec}
\usepackage{emoji}
\setemojifont{NotoColorEmoji.ttf}[Path=/usr/share/fonts/truetype/noto/]

\newtheorem{definition}{Definizione}
\newtheorem{lemma}{Lemma}
\newtheorem{theorem}{Teorema}
\newtheorem{corollary}{Corollario}

%% cambio nome al comando proof
\renewcommand*{\proofname}{Dimostrazione}

\begin{document}
\section{Io sono una prova}

Io sono del testo e rappresento un capitolo

Bello. Prova di modifica per commento-cattura


\tableofcontents

\section{Io sono una prova}

Io sono del testo e rappresento un capitolo

Bello. Prova di modifica per commento-cattura


%% Aggiungere i capitoli qui sotto
\chapter{Richiami}

\section{Spazio di Lebesgue}

\begin{definition}
    Lo \textbf{Spazio di Lebesgue} è definito come:
    
    $$
        L^1(\mathbb{R}) = \left \{ f: \mathbb{R} \rightarrow \mathbb{R} : \int_\mathbb{R} |f(t)| \ dt < +\infty \right \}
    $$
\end{definition}

Ci riferiamo anche a questo spazio come
\begin{center}
    \textit{L'insieme di tutte le funzioni sempre o assolutamente integrabili in
        modulo}.
\end{center}

\textbf{N.B.} Con la notazione "$< +\infty$" si intende "\textit{è finito}".

\begin{definition}
    Definiamo lo spazio delle funzioni sempre o assolutamente integrabili in
    modulo alla potenza $p$ come:
    
    $$
        L^p(\mathbb{R}) = \left \{ f: \mathbb{R} \rightarrow \mathbb{R} : \int_\mathbb{R} |f(t)|^p \ dt < +\infty \right \}
    $$
    
    $L^p (\mathbb{R})$ è anche detto \textbf{Spazio di Lebesgue al variare di p}
    (con $p \ge 1$).
\end{definition}

\begin{definition}
    Definiamo la \textbf{norma di $L^p$} come:
    
    $$
        ||f||_p = \left( \int_{\mathbb{R}} |f(t)|^p \ dt \right)^{\frac{1}{p}}
    $$
\end{definition}

Prendiamo ora la seguente funzione:

$$
    f : \left] 0, 1 \right] \rightarrow \mathbb{R}, \ \ f(x) = \frac{1}{\sqrt{x}}
$$

e proviamo a vedere se questa sta nello spazio $L^1(\left] 0, 1 \right])$.
\begin{equation}
    \begin{aligned}
        \int_{0}^{1} |\frac{1}{\sqrt{x}}| \ dx & = \int_{0}^{1} \frac{1}{\sqrt{x}} \ dx = \lim_{x \rightarrow 0^+} \int_{x}^{1} \frac{1}{\sqrt{t}} \ dt =          \\
                                               & = \lim_{x \rightarrow 0^+} \int_{x}^{1} t^{-\frac{1}{2}} \ dt = \lim_{x \rightarrow 0^+} [ 2 \sqrt{t} ]_{x}^{1} = \\
                                               & = \lim_{x \rightarrow 0^+} 2 [1 - \sqrt{x}] = 2 < +\infty.
    \end{aligned}
\end{equation}

Concludiamo quindi che $f(x) = \frac{1}{\sqrt{x}} \in L^1(]0, 1])$.

\paragraph{Note:}
\begin{itemize}
    \item $\int_{0}^{1} |\frac{1}{\sqrt{x}}| \ dx = \int_{0}^{1}
              \frac{1}{\sqrt{x}} \ dx$: il modulo scompare perchè
          $\frac{1}{\sqrt{x}}$ nell'intervallo $]0, 1]$ sarà sempre positivo
          !
          
          \begin{tikzpicture}
              \begin{axis}%[
                  %       xlabel=$x$, ylabel={$y$}
                  %   ]
                  \addplot {1/sqrt(x)};
              \end{axis}
          \end{tikzpicture}
\end{itemize}

\vspace{1cm}

Ora ci chiediamo se il prodotto di due funzioni che appartengono a $L^1$ sta
ancora in $L^1$. Prendiamo dunque:

$$
    f(x) \cdot f(x) = \frac{1}{x}
$$

e verifichiamo se $\frac{1}{x} \in L^1(]0, 1])$:

\begin{equation}
    \begin{aligned}
        \int_{0}^{1} \frac{1}{x} \ dx & = \lim_{x \rightarrow 0^+} \int_{x}^{1} \frac{1}{t} \ dt = \lim_{x \rightarrow 0^+} [log |t|]_{x}^{1} = \\
                                      & = \lim_{x \rightarrow 0^+} [log 1 - log|x|] = +\infty.
    \end{aligned}
\end{equation}

Deduciamo quindi che non sempre il prodotto di due funzioni che stanno in $L^1$
appartiene a $L^1$.

\paragraph{Note:}
\begin{itemize}
    \item Ricordiamo che $log 1 = 0$
    \item Ricordiamo che $log |x|$ con $x$ che tende a $0^+$ (tende a 0 dalla
          destra) va a $-\infty$
          
          \begin{tikzpicture}
              \begin{axis}%[
                  %       xlabel=$x$, ylabel={$y$}
                  %   ]
                  \addplot {ln(x)};
              \end{axis}
          \end{tikzpicture}
    \item $\lim_{x \rightarrow 0^+} [log 1 - log|x|] = 0 - (-\infty) = +\infty$
\end{itemize}

\vspace{1cm}

Per far sì che il prodotto appartenga a $L^1$ dobbiamo introdurre il concetto di
\textit{\textbf{Prodotto di Convoluzione}}.

\section{Prodotto di Convoluzione}
Il \textbf{Prodotto di Convoluzione} è importante perchè ci permette di
mantenere l'appartenenza allo spazio della singola immagine. Questo, può essere
espresso, in modo formale, nel seguente modo:
\begin{center}
    La \textit{Convoluzione} è una tecnica che consente di regolarizzare le
    funzioni e di approssimarle in $L^p$.
\end{center}

\begin{definition}
    Siano $f \in L^1(\mathbb{R})$ e $g \in L^p(\mathbb{R})$ con $1 \leq p \leq
        +\infty$. \'E possibile definire il prodotto di convoluzione tra $f$ e
    $g$ come:
    
    $$
        \left( f \star g\right)(x) = \int_{\mathbb{R}} f(x - y) g(y) \ dy
    $$
    
    e
    
    $$
        ||f \star g||_p \leq ||f||_1 \cdot ||g||_p.
    $$
    
    Ovvero, il Prodotto di Convoluzione sta in $L^p (\mathbb{R})$
    
    $$f \star g \in L^p(\mathbb{R})
    $$
    
    Risulta quindi vero che $||f \star g||_p < +\infty$.
    
\end{definition}

\vspace{0.3cm}

\textbf{N.B.} Il Prodotto di Convoluzione eredita le migliori proprietà delle due
funzioni.

\vspace{1cm}

Diamo ora il seguente lemma riassuntivo che raccoglie tutte le principali
proprietà del prodotto di convoluzione.

\begin{lemma}
    Date $f, g, h \in L^1(\mathbb{R})$, risulta:
    
    \begin{enumerate}
        \item $f \star g = g \star f$ (proprietà commutativa);
        \item $f \star (g + h) = (f \star g) + (f \star h)$ (proprietà
              distributiva rispetto alla somma);
        \item $(f \star g) \star h = f \star (g \star h)$ (proprietà
              associativa);
        \item posto $\tau_a f(x) := f(x + a), \ x \in \mathbb{R}$ (operatore di
              traslazione,) risulta:
              $$
                  \tau_a (f \star g) = (\tau_a f) \star g
              $$
              (invarianza per traslazioni).
    \end{enumerate}
\end{lemma}


\section{Trasformata di Fourier}

In questa sezione daremo la definizione e richiameremo alcune importanti
proprietà che riguardano la trasformata di Fourier.\\
La trasformata di Fourier viene introdotta per poter dare una rappresentazione
simile a quella fornita dalle serie di Fourier per le funzioni non periodiche.

\begin{definition}
    % TODO: aggiungere ?? f (una funzione che rappresenta un segnale) 
    Sia $f \in L^1(\mathbb{R})$. Allora la trasformata di Fourier di $f$ è
    definita come:
    
    $$
        \hat{f}(\lambda) := \int_{-\infty}^{+\infty} f(x) e^{-i\lambda x} \ dx,
        \ \lambda \in \mathbb{R}
    $$
\end{definition}

\paragraph{Note:}
\begin{itemize}
    \item Ricordiamo che $e^{-i\lambda x} = \cos(\lambda x) - i \sin(\lambda x)$
    \item Ricordiamo i numeri complessi definiti come:
          $$
              z = a + ib
          $$
          dove $a$ viene detta \textit{parte intera} e $b$ la \textit{parte
              immaginaria}.
    \item Il modulo di un numero complesso è definito come:
          $$
              |z| = \sqrt{a^2 + b^2}
          $$
          \begin{center}
              \begin{tikzpicture}[x=1.0pt,y=1.0pt,yscale=-1,xscale=1]
                  %uncomment if require: \path (0,300); %set diagram left start
                  %at 0, and has height of 300
                  
                  %Straight Lines [id:da5730343469684522] 
                  \draw    (150.5,189.75) -- (150.5,234.75) ;
                  %Straight Lines [id:da70665287162995] 
                  \draw    (99.5,189.75) -- (150.5,189.75) ;
                  %Shape: Axis 2D [id:dp7340131235297938] 
                  \draw  (90,235) -- (190,235)(100,145) -- (100,245) (183,230)
                  -- (190,235) -- (183,240) (95,152) -- (100,145) -- (105,152)
                  ;
                  %Shape: Circle [id:dp17035889321633024] 
                  \draw  [fill={rgb, 255:red, 0; green, 0; blue, 0 }  ,fill
                      opacity=1 ] (148,189.25) .. controls (148,187.87) and
                  (149.12,186.75) .. (150.5,186.75) .. controls (151.88,186.75)
                  and (153,187.87) .. (153,189.25) .. controls (153,190.63) and
                  (151.88,191.75) .. (150.5,191.75) .. controls (149.12,191.75)
                  and (148,190.63) .. (148,189.25) -- cycle ;
                  %Straight Lines [id:da11421787184111976] 
                  \draw  [dash pattern={on 0.84pt off 2.51pt}]  (150.5,189.25)
                  -- (100,235) ;
                  
                  % Text Node
                  \draw (154,165) node [anchor=north west][inner sep=0.75pt]
                  [align=left] {z};
                  % Text Node
                  \draw (145,239) node [anchor=north west][inner sep=0.75pt]
                  [align=left] {a};
                  % Text Node
                  \draw (84.5,182) node [anchor=north west][inner sep=0.75pt]
                  [align=left] {b};
                  % Text Node
                  \draw (106,196) node [anchor=north west][inner sep=0.75pt]
                  [align=left] {|z|};
                  
                  
              \end{tikzpicture}
          \end{center}
    \item Quindi
          $$
              |e^{-i \lambda x}| = (\cos^2 (\lambda x) + \sin^2(\lambda
              x))^{\frac{1}{2}} = 1
          $$
          
\end{itemize}

\vspace{1cm}

Ci domandiamo il perché sia così importante che $f \in L^1(\mathbb{R})$
(possibile domanda di esame).\\

\begin{center}
    Se $f \in L^1(\mathbb{R})$ allora $\hat{f} \in L^1(\mathbb{R})$.
\end{center}

\begin{proof}
    Poiché $f \in L^1(\mathbb{R})$, la funzione $\hat{f}(\lambda)$ risulterà
    sicuramente ben definita, infatti:
    
    \begin{equation}
        \begin{aligned}
            |\hat{f}(\lambda)| & = \left|\int_{\mathbb{R}} f(t) e^{-i \lambda t} \ dt \right| \leq \int_{\mathbb{R}} |f(t) e^{-i \lambda t}| \ dt =                  \\
                               & = \int_{\mathbb{R}} |f(t)| \cdot |e^{-i \lambda t}| \ dt = \int_{\mathbb{R}} |f(t)| \ dt < + \infty, \forall \lambda \in \mathbb{R}
        \end{aligned}
    \end{equation}
    
    Risulta essere $< +\infty$ \textit{se e solo se} $f\in L^1(\mathbb{R})$
    
\end{proof}

%%siamo sicuri di questa cosa ???
Ma, in generale non possiamo affermare che $\hat{f}(\lambda)
    \in L^1(\mathbb{R})$.\\

%% TODO: la successiva parte che riguarda il grafico del segnale non l'ho
%capita.

Ci accorgiamo ora che, nella realtà, molto spesso abbiamo a che fare con segnali
\textit{"in banda"} (espressi con la rispettiva trasformata di Fourier) e
risulterebbe molto comodo poter riuscire a risalire alla funzione originale. In
altre parole ci domandiamo se esiste l'inverso della Trasformata di Fourier.

\begin{theorem}
    Sia $f \in L^1(\mathbb{R}) \cap C^0(\mathbb{R})$ e tale che $\hat{f} \in
        L^1(\mathbb{R})$. Allora definiamo la \textit{Trasformata Inversa di
        Fourier} come:
    
    $$
        f(x) = \int_{-\infty}^{+\infty} \hat{f}(\lambda) e^{i \lambda x} \
        d\lambda, \ \ \forall x \in \mathbb{R}
    $$
\end{theorem}

\begin{proof}
    Poiché $\hat{f} \in L^1(\mathbb{R})$, la funzione $f(x)$ risulterà
    sicuramente ben definita, infatti:
    
    \begin{equation}
        \begin{aligned}
            |f(x)| & = \left|\int_{\mathbb{R}} \hat{f}(\lambda) e^{i \lambda t} \ d\lambda \right| \leq \int_{\mathbb{R}} |\hat{f}(\lambda)| \ d\lambda < + \infty, \forall x \in \mathbb{R}
        \end{aligned}
    \end{equation}
    
    Risulta essere $< +\infty$ \textit{se e solo se} $\hat{f}\in L^1(\mathbb{R})$
    
\end{proof}

Il precedente Teorema permette di esprimere $f$ in termini della sua Trasformata
di Fourier. Questo rappresenta la Trasformata Inversa di Fourier e possiamo
indicarla anche con $\hat{f}^{-1}$.

\begin{center}

    \tikzset{every picture/.style={line width=0.75pt}}
    
    \begin{tikzpicture}[x=0.75pt,y=0.75pt,yscale=-1,xscale=1]
        %uncomment if require: \path (0,300); %set diagram left start at 0, and
        %has height of 300
        
        %Shape: Axis 2D [id:dp5823155267563114] 
        \draw  (88,155.67) -- (188,155.67)(138.33,71) -- (138.33,171)
        (181,150.67) -- (188,155.67) -- (181,160.67) (133.33,78) -- (138.33,71)
        -- (143.33,78)  ;
        %Shape: Axis 2D [id:dp9473219631031637] 
        \draw  (400,155.67) -- (500,155.67)(450.33,71) -- (450.33,171)
        (493,150.67) -- (500,155.67) -- (493,160.67) (445.33,78) -- (450.33,71)
        -- (455.33,78)  ;
        %Curve Lines [id:da9507777176571728] 
        \draw    (90,129) .. controls (130,99) and (150,159) .. (190,129) ;
        %Curve Lines [id:da8695662202477725] 
        \draw    (398.67,142) .. controls (434.33,143) and (436.33,105) ..
        (451,104.33) .. controls (465.67,103.67) and (453.67,141.67) ..
        (498.67,142) ;
        %Curve Lines [id:da04668467081792005] 
        \draw [color={rgb, 255:red, 208; green, 2; blue, 27 }  ,draw opacity=1 ]
        (216.33,69) .. controls (255.13,39.9) and (349.45,46.87) ..
        (374.84,68.3) ; \draw [shift={(377,70.33)}, rotate = 226.71] [fill={rgb,
                    255:red, 208; green, 2; blue, 27 }  ,fill opacity=1 ][line width=0.08]
        [draw opacity=0] (8.93,-4.29) -- (0,0) -- (8.93,4.29) -- cycle    ;
        %Curve Lines [id:da6399644807337441] 
        \draw [color={rgb, 255:red, 208; green, 2; blue, 27 }  ,draw opacity=1 ]
        (220.96,189.96) .. controls (244.28,247.55) and (372.49,228.17) ..
        (380.33,187.67) ; \draw [shift={(219.67,186.33)}, rotate = 72.68]
        [fill={rgb, 255:red, 208; green, 2; blue, 27 }  ,fill opacity=1 ][line
            width=0.08]  [draw opacity=0] (8.93,-4.29) -- (0,0) -- (8.93,4.29) --
        cycle    ;
        
        % Text Node
        \draw (287.33,56) node [anchor=north west][inner sep=0.75pt]
        [color={rgb, 255:red, 208; green, 2; blue, 27 }  ,opacity=1 ]
        [align=left] {$\displaystyle \hat{f}$};
        % Text Node
        \draw (289.33,191.33) node [anchor=north west][inner sep=0.75pt]
        [color={rgb, 255:red, 208; green, 2; blue, 27 }  ,opacity=1 ]
        [align=left] {$\displaystyle f$};
        
        
    \end{tikzpicture}
    
\end{center}

\paragraph{Note:}
\begin{itemize}
    \item $C^0(\mathbb{R}) = \{ f: \mathbb{R} \rightarrow \mathbb{R} | f \text{
                  risulta continua in } \mathbb{R} \}$. Ovvero, $C^0(\mathbb{R})$ è
          l'insieme di tutte le funzioni che risultano continue in
          $\mathbb{R}$.
\end{itemize}

Analizziamo ora alcune proprietà della Trasformata di Fourier.

\begin{theorem}
    Sia $f \in L^1(\mathbb{R})$, allora la funzione $\hat{f}(\lambda) \in
        C^0(\mathbb{R})$ e vale:
    
    $$
        \lim_{\lambda \rightarrow +\infty} \left| \hat{f}(\lambda) \right| = 0.
    $$
\end{theorem}

Ricordiamo che le funzioni che sono in $L^1(\mathbb{R})$ non è detto che siano
continue ! Una funzione può essere integrabile anche se ha un numero finito di
punti di discontinuità. Il precedente teorema ci va a dire che se applico la
Trasformata di Fourier ad una funzione che non è continua (ma comunque
integrabile), allora il risultato sarà sicuramente una funzione continua e
che quella funzione sarà integrabile ! Questo è anche visibile dal grafico che
otterremmo in quanto avrà le "code" che si schiacciano su zero.\\
Questa proprietà viene anche chiamata \textit{"Regolarizzazione"}.
%% TODO: c'è anche un secondo significato a questo teorema, ci assicura che %
%all'infinito la funzione tende a 0, ma non so bene spiegarlo. Vanno rivisti gli
%appunti (ci assicura che sia 0 perchè se no si schiacciasse l'area sotto la
%funzione tenderebbge a +inf, ed il suo itegrale sarebbe di conseguenza non finito)

\begin{theorem}
    Siano $f, g \in L^1(\mathbb{R})$ e $\alpha, \beta \in \mathbb{C}$ allora:
    
    $$
        (\widehat{ \alpha f + \beta g })(\lambda) = \alpha \hat{f}(\lambda) +
        \beta \hat{g}(\lambda).
    $$
\end{theorem}

Questo teorema ci dice che la Trasformata di Fourier è anche un operatore
\textbf{\textit{Lineare}}.\\

Ci domandiamo ora se esiste qualche legame tra la Trasformata di Fourier e la
sua Derivata Prima e se c'è la possibilità di legarle in qualche modo.

\begin{theorem}
    Sia $f \in L^1(\mathbb{R}) \cap C^1(\mathbb{R})$ tale che $f^{\prime} \in
        L^1(\mathbb{R})$. Allora:
    
    $$
        \widehat{f^{\prime}}(\lambda) = i \lambda \hat{f}(\lambda), \ \ \forall
        \lambda \in \mathbb{R}.
    $$
\end{theorem}

\paragraph{Note:}
\begin{itemize}
    \item $C^1(\mathbb{R}) = \{ f: \mathbb{R} \rightarrow \mathbb{R} \ | \
              \exists f^{\prime} \text{ continua in } \mathbb{R} \}$. Questo è
          l'insieme delle funzioni la cui Derivata Prima è continua in
          $\mathbb{R}$.
\end{itemize}

Chiaramente, il risultato del precedente Teorema può essere generalizzato per le
derivate di ordine superiore. Infatti:

\begin{theorem}
    Sia $f \in L^1(\mathbb{R}) \cap C^k(\mathbb{R})$ con $k \in \mathbb{N}$, $k
        \geq 2$ e $f^{\left(j\right)} \in L^1(\mathbb{R})$ per ogni $j = 1, 2,
        \ldots, k$, allora risulta banalmente che:
    
    $$
        \widehat{f^k}(\lambda) = (i \lambda)^k \hat{f}(\lambda), \ \ \lambda \in
        \mathbb{R}.
    $$
\end{theorem}

\paragraph{Note:}
\begin{itemize}
    \item $C^k(\mathbb{R}) = \{ f: \mathbb{R} \rightarrow \mathbb{R} \ | \
              \exists f^{k} \text{ continua in } \mathbb{R} \}$. Questo è
          l'insieme delle funzioni la cui Derivata k-esima e di conseguenza tutte le precedenti sono continue in
          $\mathbb{R}$.
\end{itemize}

Vale il seguente corollario.

%% TODO: ricontrollare, differenze tra Vinti e Fagiolo.
\begin{corollary}
    Sia $f \in L^1(\mathbb{R}) \cap C^k(\mathbb{R})$ e $f^{\left(j\right)} \in
        L^1(\mathbb{R})$ per ogni $j = 1, 2, \ldots, k$. Allora:
    
    $$
        \lim_{ |\lambda| \rightarrow +\infty } \left| \hat{f}^k(\lambda) \right| = 0 \rightarrow
        \lim_{ |\lambda| \rightarrow +\infty } \left|\lambda\right|^k\left| \hat{f}(\lambda) \right| = 0
    $$
\end{corollary}

Applicando questo corollario ad $f \in L^1(\mathbb{R}) \cap C^2(\mathbb{R})$ si ha che:

    $$
        \lim_{\lambda \rightarrow \infty} \left| \lambda \right|^2 \left|
        \hat{f}(\lambda) \right|  = \lim_{\lambda \rightarrow \infty}
        \frac{\left|\hat{f}(\lambda)\right|}{ \frac{1}{ \left| \lambda
                \right|^2}} = 0.
    $$

Questo ci dice che $\left|\hat{f}(\lambda)\right|$ è un infinitesimo più veloce di
$\left|\lambda\right|^2$, per $\lambda$ che tende a infinito, perciò sappiamo che $\hat{f}$ è
assolutamente integrabile. Quindi $\hat{f} \in L^1(\mathbb{R})$. Questo
corollario ci fornisce anche una condizione sufficiente per l'integrabilità di
$\hat{f}(\lambda)$.

\begin{theorem}
    Sia $f \in L^1(\mathbb{R}) \cap C^0(\mathbb{R})$.
    Se $\hat{f}(\lambda) = 0, \forall \lambda \in \mathbb{R}$, allora $f(x) = 0, \forall x \in \mathbb{R}$.\\
\end{theorem}

\begin{corollary}
    Se due funzioni $f, g \in L^1(\mathbb{R}) \cap C^0(\mathbb{R})$ e $\hat{f}(\lambda) = \hat{g}(\lambda), \forall \lambda \in \mathbb{R}$,
    allora $f = g$ in $\mathbb{R}$.
\end{corollary}

\begin{theorem}
    Siano $f, g \in L^1(R)$. Allora,
    
    $$
        \widehat{f \star g}(\lambda) = \hat{f}(\lambda) \hat{g}(\lambda), \
        \forall \lambda \in \mathbb{R}.
    $$
\end{theorem}

Il precedente teorema ci mostra che la Trasformata di Fourier del prodotto di
convoluzione è uguale al prodotto delle trasformate. Questo è un importante teorema
perché, a livello computazionale, è molto costoso computare la trasformata di Fourier
del prodotto di convoluzione, ma è molto più semplice effettuare una banale moltiplicazione
tra due funzioni.

%% TODO: C'è una sezione nelle dispense di Vinti riguardante
%% la storia del campionamento. Non so se metterla 🤔
\section{Teoria del Campionamento}

\subsection{Storia del Campionamento}
\TODO[inline]{Completare}

\subsection{Teorema del Campionamento}

Diamo la definizione del classico \textit{WKS-sampling theorem} o
\textit{Teorema del Campionamento}. Definiamo prima il \textbf{supporto}
di una funzione $f$ come:

$$
    \sup f = \{ x \in \mathbb{R} : f(x) \neq 0 \}.
$$

%% TODO: qui ci stanno alcune discrepanze tra Vinti e Fagiolo, va ricontrollato.
\begin{theorem}
    Sia $f \in L^2(\mathbb{R}) \cap C(\mathbb{R})$ una funzione continua, ad energia finita e
    a banda limitata, i.e., tale che
    $\sup \hat{f} \subset \left[ -\pi W, \ \pi W\right], W > 0$.
    Allora vale la seguente formula di interpolazione.

    $$
        f(t) = \sum_{k = - \infty}^{+\infty} f \left(\frac{k}{W}\right) sinc \left[ \pi(Wt - k) \right], \text{ con } t \in \mathbb{R}
    $$

    dove:

    $$
        sinc(t) :=
        \left\{ \begin{array}{cl}
            \frac{sin(\pi t)}{\pi t} & \ t \neq 0 \\
            1                        & \ t = 0
        \end{array} \right.
    $$
\end{theorem}


\begin{figure}[H]
    \centering
    \begin{tikzpicture}[x=0.75pt,y=0.75pt,yscale=-1,xscale=1]
        %uncomment if require: \path (0,300); %set diagram left start at 0, and has height of 300

        %Shape: Axis 2D [id:dp6614397845308091] 
        \draw  (204,183.5) -- (379,183.5)(291,82) -- (291,196.5) (372,178.5) -- (379,183.5) -- (372,188.5) (286,89) -- (291,82) -- (296,89)  ;
        %Straight Lines [id:da7696580538613964] 
        \draw    (250,161.25) -- (250,183.25) ;
        %Straight Lines [id:da059819649685002974] 
        \draw    (331,161.25) -- (331,183.25) ;
        %Straight Lines [id:da32947787237217474] 
        \draw    (331,161.25) -- (250,161.25) ;

        % Text Node
        \draw (222.5,191.5) node [anchor=north west][inner sep=0.75pt]   [align=left] {$\displaystyle -\pi W$};
        % Text Node
        \draw (316,192) node [anchor=north west][inner sep=0.75pt]   [align=left] {$\displaystyle \pi W$};
    \end{tikzpicture}
    \caption{Una funzione a banda limitata}
\end{figure}

\begin{figure}[H]
    \centering
    \begin{tikzpicture}[x=0.75pt,y=0.75pt,yscale=-1,xscale=1]
        %uncomment if require: \path (0,300); %set diagram left start at 0, and has height of 300

        %Shape: Axis 2D [id:dp6614397845308091] 
        \draw  (204,230.75) -- (542.5,230.75)(306,82) -- (306,250.25) (535.5,225.75) -- (542.5,230.75) -- (535.5,235.75) (301,89) -- (306,82) -- (311,89)  ;
        %Straight Lines [id:da059819649685002974] 
        \draw    (421,226.75) -- (421,237.25) ;
        %Straight Lines [id:da333995992515034] 
        \draw    (441,226.75) -- (441,237.25) ;
        %Straight Lines [id:da4908605908983048] 
        \draw    (461,226.75) -- (461,237.25) ;
        %Straight Lines [id:da5180908391730039] 
        \draw    (400,226.75) -- (400,237.25) ;
        %Curve Lines [id:da6657246696267412] 
        \draw    (254,118.5) .. controls (269.59,106.81) and (338.93,97.87) .. (337.5,120.75) .. controls (336.07,143.63) and (365.6,210.17) .. (385,212.25) .. controls (404.4,214.33) and (425,145.75) .. (443,135.25) .. controls (461,124.75) and (493.73,190.83) .. (506.5,181.25) ;
        %Shape: Circle [id:dp878091919683829] 
        \draw  [color={rgb, 255:red, 208; green, 2; blue, 27 }  ,draw opacity=1 ][fill={rgb, 255:red, 208; green, 2; blue, 27 }  ,fill opacity=1 ] (397,202) .. controls (397,200.62) and (398.12,199.5) .. (399.5,199.5) .. controls (400.88,199.5) and (402,200.62) .. (402,202) .. controls (402,203.38) and (400.88,204.5) .. (399.5,204.5) .. controls (398.12,204.5) and (397,203.38) .. (397,202) -- cycle ;
        %Shape: Circle [id:dp017384754588354534] 
        \draw  [color={rgb, 255:red, 208; green, 2; blue, 27 }  ,draw opacity=1 ][fill={rgb, 255:red, 208; green, 2; blue, 27 }  ,fill opacity=1 ] (417,169.5) .. controls (417,168.12) and (418.12,167) .. (419.5,167) .. controls (420.88,167) and (422,168.12) .. (422,169.5) .. controls (422,170.88) and (420.88,172) .. (419.5,172) .. controls (418.12,172) and (417,170.88) .. (417,169.5) -- cycle ;
        %Shape: Circle [id:dp3580711979696489] 
        \draw  [color={rgb, 255:red, 208; green, 2; blue, 27 }  ,draw opacity=1 ][fill={rgb, 255:red, 208; green, 2; blue, 27 }  ,fill opacity=1 ] (436,139.5) .. controls (436,138.12) and (437.12,137) .. (438.5,137) .. controls (439.88,137) and (441,138.12) .. (441,139.5) .. controls (441,140.88) and (439.88,142) .. (438.5,142) .. controls (437.12,142) and (436,140.88) .. (436,139.5) -- cycle ;
        %Shape: Circle [id:dp2482434926915713] 
        \draw  [color={rgb, 255:red, 208; green, 2; blue, 27 }  ,draw opacity=1 ][fill={rgb, 255:red, 208; green, 2; blue, 27 }  ,fill opacity=1 ] (458,141.5) .. controls (458,140.12) and (459.12,139) .. (460.5,139) .. controls (461.88,139) and (463,140.12) .. (463,141.5) .. controls (463,142.88) and (461.88,144) .. (460.5,144) .. controls (459.12,144) and (458,142.88) .. (458,141.5) -- cycle ;

        % Text Node
        \draw (412.5,243) node [anchor=north west][inner sep=0.75pt]  [font=\scriptsize] [align=left] {$\displaystyle \frac{k}{W}$};
        % Text Node
        \draw (435,243) node [anchor=north west][inner sep=0.75pt]  [font=\scriptsize] [align=left] {$\displaystyle \frac{k}{W}$};
    \end{tikzpicture}
    \caption{Esempio di applicazione del campionamento}
\end{figure}

In altre parole, il teorema del campionamento ci dice che, da una famiglia discreta di
valori campione, assunti in modo uniformemente spaziato su tutto l’asse reale, è possibile
ricostruire in modo esatto, un segnale ad energia finita (ovvero in $L^2(\mathbb{R})$) e a
banda limitata, mediante la serie interpolante introdotta sopra.\\

$\Delta t = 1/W$ denota l’\textit{intervallo temporale} tra i campioni, $\pi W/ 2\pi = w/2$ denota
l’\textit{ampiezza di banda} di $f$, che e misurata in cicli per unita di tempo (Hertz) e
$R = 1 /\Delta t$ denota l’\textit{ordine di campionamento} (\textit{sampling rate}),
misurato in campioni per unita di tempo.

H. Nyquist mise in rilievo il significato dell’“intervallo”(numero) $1/W$ nella telegrafia
e Shannon lo chiamo \textit{intervallo di Nyquist} corrispondente alla frequenza di
banda $[−\pi W, \pi W]$. L’ordine di campionamento $R$, legato all’intervallo di Nyquist,
rappresenta un minimo teorico per la ricostruzione del segnale.
Per essere più precisi, il teorema sampling costituisce un algoritmo che permette di
rispondere alle due seguenti domande:

\begin{enumerate}
    \item \textit{Quanti campioni sono necessari per assicurare che l’informazione
              contenuta nel segnale venga preservata, durante il processo di ricostruzione?}\\

          Se il segnale contiene componenti di alte frequenze, è necessario campionarlo almeno
          due volte la massima frequenza del segnale per evitare perdite di informazione,
          cioe $f_s \ge 2 f_H$ , dove $f_s$ e la frequenza sampling, $f_H$ e la piu alta
          frequenza del segnale e $f_s = 2 f_H$ è la frequenza (ordine) di Nyquist.
          Campionare ad un ordine più alto di $2 f_H$ significa utilizzare una successione
          più fine di valori campione, e questo è il caso del cosiddetto sovracampionamento (oversampling).
          Nella pratica il sovracampionamento deve essere effettuato molto spesso, poichè
          un fattore correttivo deve essere introdotto per compensare il fatto che il
          campionamento e l’interpolazione non possono coincidere con i corrispondenti valori
          teorici; questo è il caso del cosiddetto \textit{round-off error} o \textit{errore di
              quantizzazione}. Inoltre un altro errore temporale si presenta quando i campioni
          non possono essere presi esattamente agli istanti temporali $k/W$; questo errore è
          chiamato \textit{time-jitter error}.
          Un esempio concreto di sovracampionamento è rappresentato dal un lettore di compact disk.
          Infatti considerando che la piu alta frequenza
          udibile è approssimativamente intorno a 18.000 cicli al secondo (Hertz),
          a seconda dell’ascoltatore, in base alla teoria precedente un segnale audio
          deve essere campionato almeno 36.000 volte al secondo per far sì che
          il segnale venga ricostruito completamente. L’ordine di campionamento
          dei primi lettori di compact disk era di circa 44.000 campioni al secondo.
          Altre volte capita di non avere a disposizione un numero ragionevole di valori
          campione per ricostruire il segnale completamente. Questo
          avviene quando non conosciamo il segnale originale, ma dobbiamo ricostruirlo dalle
          informazioni che abbiamo a disposizione del segnale (cioè
          dai suoi valori campione). In questo caso, cioè quando il numero di
          informazioni (valori campione) è scarso, la distanza tra i nodi è più
          grande dell’intervallo di Nyquist ed abbiamo il cosiddetto sottocampionamento
          (undersampling). In questo caso si presenta il fenomeno di
          aliasing e sostanzialmente cio che succede è che le replicanti spettrali si
          sovrappongono.

          \begin{figure}[H]
              \centering
              \includegraphics[width=8cm, keepaspectratio]{capitoli/immagini/imgs/aliasing_tajmahal.jpg}
              \caption{Nella figura possiamo apprezzare un esempio di aliasing.}
          \end{figure}

    \item \textit{Dato un segnale, campionato ad un certo fissato ordine di campionamento,
              quali frequenze non devono essere contenute nel segnale al fine di preservare
              l’informazione?}\\

          Questa è la situazione inversa rispetto a quella presentata nella prima
          domanda. La risposta è che un segnale campionato ad un fissato ordine
          di campionamento non può contenere componenti a frequenze superiori
          alla metà dell’ordine di campionamento, cioe: $f_H \leq f_s/2$, dove
          $f_H$ è la frequenza più alta contenuta nel segnale e $f_s$ è la frequenza
          sampling. In altre parole, la più alta frequenza che puo essere accuratamente
          rappresentata è la metà della frequenza sampling. Quindi il teorema
          sampling fornisce un limite superiore per la più alta frequenza $f_H$.
          Spieghiamolo con un esempio: supponiamo di utilizzare un CD-rom con un
          fissato ordine di campionamento di 44.1KHz. Nyquist afferma che la più
          alta frequenza che può essere riprodotta da questo campionamento è di
          22.05KHz. Infatti le frequenze superiori a 22.05KHz producono aliasing.
          Per evitare questo, può essere utilizzato un filtro passa basso per bloccare
          le frequenze superiori a 22.05KHz. È importante notare che l’aliasing
          può essere prodotto sia da segnali campionati non correttamente (con
          un numero di campioni non sufficiente) che da frequenze troppo alte del
          segnale aventi un fissato ordine di campionamento.
\end{enumerate}

\subsection{Svantaggi}

Benchè il teorema del campionamento venga laramente impiegato in ambito applicativo
esso presenta svantaggi o limiti da questo punto di vista.

\begin{itemize}
    \item Le ipotesi fatte sul segnare da ricostruire $f$, ovvero che sia ad energia finita
          e a banda limitata, implicano che ci riduciamo a considerare una famiglia di segnali
          estremamente regolari e ci consente di ricostruire solo sengali appartenenti alla classe
          $C^{\infty}(\mathbb{R})$. Nelle applicazioni reali, i segnali difficilmente sono così
          regolari. Per esempio, nelle immagini, forti salti di livello nella scala di grigi
          (vedi bordi degli oggetti), matematicamente possono essere visti come delle
          discontinuità.
    \item Per ricostruire un segnale $f$ al tempo $t$ è necessario un numero di campioni infinito
          e il loro valore viene rilevato sia all'istante precedente a $t$ che a quelli successivi.
          Dal punto di vista applicativo, una prima difficoltà potrebbe essere quella di
          implementare una formula teorica che prevede infiniti termini.
          Il problema potrebbe essere risolto troncando la serie; tuttavia, questo introdurrebbe
          un errore (di troncamento) nella ricostruzione, che non garantirebbe piu
          l’interpolazione del segnale ma una mera approssimazione.
          Inoltre, nei problemi reali, i segnali sono noti solo nel “passato”, ovvero i
          valori campione di $f$ sono noti, in generale, solo negli istanti di tempo $k/W$
          precedenti a $t$. I valori di $f(k/W)$ nel “futuro”, ovvero quando $k/W > t$,
          non sono in generale disponibili.
    \item Per il principio di Hisemberg i segnali non possono essere contemporanemanete
          a durata e banda limitata, quindi i segnali nel mondo reale, in generale, non soddisfano
          le ipotesi del teorema del campionamento.
\end{itemize}

\subsection{Operatore Sampling Generalizzato}

Per poter rendere applicabile il teorema del campionamento ed evitare le
problematiche viste in precedenza, è stato studiato un \textit{operatore di tipo
    Sampling Generalizzato}, con lo scopo di fornire una versione approssimata del
teorema. L'idea di base è quella di sostituire la funzione $sinc$ con una
funzione $\varphi$ a supporto continuo e compatto, il che implica che la serie
si riduce ad una somma finita per ogni $t \in \mathbb{R}$ e dunque, per ricostruire
un dato segnale in un certo istante di tempo, sono necessari solo un numero
finito di valori campione.\\

Ora per dimostrare che i campioni possono essere assunti nel passato solo rispetto
a $t$ forniremo due definizioni:

\begin{definition}
    Sia $\varphi \in C_c(\mathbb{R})$ e $f \in C(\mathbb{R})$, definiamo la famiglia
    di operatori $(S_W^{\varphi}f)_{W>0}$ con

    $$
        (S_W^{\varphi}f)(t) := \sum_{k = - \infty}^{+\infty} f(\frac{k}{W}) \varphi(Wt-k), \ t \in \mathbb{R}
    $$
\end{definition}

L'operatore $S_W^{\varphi}$ è detto \textit{Operatore Sampling Generalizzato}, mentre
$\varphi$ è detta \textit{Nucleo} dell'operatore.

\begin{definition}
    Data la funzione $\varphi: \ \mathbb{R} \rightarrow \mathbb{R}$ e un numero
    $\beta > 0$, possiamo definire

    $$
        m_\beta(\varphi) := \sup_{u \in \mathbb{R}} \sum_{k \in \mathbb{Z}} |u-k|^\beta |\varphi(u-k)|
    $$

    che prende il nome di \textit{Momento assoluto discreto} di ordine $\beta$ della funzione
    $\varphi$.
\end{definition}

Grazie alla definizione appena introdotta otteniamo la seguente stima:

$$
    |(S_W^{\varphi}f)(t)| \leq \sum_{k = - \infty}^{+\infty} |f(\frac{k}{W})| |\varphi(Wt-k)| \leq ||f||_{\infty} \sum_{k = - \infty}^{+\infty} |\varphi(Wt-k)| \leq ||f||_{\infty} m_0(\varphi)
$$

$\forall t \in \mathbb{R}$.\\

Poichè $f$ è limitata e il momento $m_0(\varphi) \leq +\infty$ risulta che
l'operatore di sampling generalizzato risulta ben definito in $C(\mathbb{R})$.
Vale quindi il seguente risultato di approssimazione.

\begin{theorem}
    Sia $\varphi \in C_c(\mathbb{R})$ tale che

    \begin{equation}\label{eq:uno}
        \sum_{k = - \infty}^{+\infty} \varphi(u-k) = 1, \ \forall u \in \mathbb{R}
    \end{equation}

    Allora, se la funzione limitata $f: \mathbb{R} \rightarrow \mathbb{R}$ è continua
    in $t_0 \in \mathbb{R}$, risulta:

    $$
        \lim_{W \rightarrow \infty} (S_W^{\varphi}f)(t_0) = f(t_0)
    $$

    In particolare, se $f \in C(\mathbb{R})$ si ha:

    $$
        \lim_{W \rightarrow \infty} ||S_W^{\varphi}f - f||_\infty = 0
    $$
\end{theorem}

Da qui deriva il corollario che ci mostra com'è possibile effettuare un processo
di ricostruzione grazie a campioni rilevati esclusivamente nel passato.

\begin{corollary}
    Sia $\varphi \in C_c(\mathbb{R})$, con $\sup \varphi \subset \mathbb{R}^+$ e
    tale che sia soddisfatta la condizione (\ref{eq:uno}), allora per ogni funzione limitata
    $f: \mathbb{R} \rightarrow \mathbb{R}$, nei punti $t_0$ di continuità si ha:

    $$
        \lim_{W \rightarrow \infty} (S_W^{\varphi}f)(t_0) = \lim_{W \rightarrow \infty} \sum_{\frac{k}{W} < t_0} f(\frac{k}{W}) \varphi(Wt_0 - k) = f(t_0)
    $$
\end{corollary}

\begin{proof}
    Come conseguenza del fatto che $\varphi(Wt_0-k) = 0$ per $\frac{k}{W} \ge t_0$ risulta che

    $$
        (S_W^{\varphi}f)(t_0) = \sum_{\frac{k}{W} < t_0} f(\frac{k}{W}) \varphi(Wt_0 - k)
    $$

    Quindi la tesi segue banalmente per il teorema precedente.
\end{proof}

\paragraph{Note}
\begin{itemize}
    \item $\varphi$ si legge \textit{Phi}.
    \item $C_c(\mathbb{R})$ indica l'insieme delle funzioni continue a supporto
          compatto.
\end{itemize}
\section{Io sono una prova}

Io sono del testo e rappresento un capitolo

Bello. Prova di modifica per commento-cattura


\end{document}
