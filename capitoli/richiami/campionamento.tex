%% TODO: C'è una sezione nelle dispense di Vinti riguardante
%% la storia del campionamento. Non so se metterla 🤔
\section{Teoria del Campionamento}

\subsection{Storia del Campionamento}
\TODO[inline]{Completare}

\subsection{Teorema del Campionamento}

Diamo la definizione del classico \textit{WKS-sampling theorem} o
\textit{Teorema del Campionamento}. Definiamo prima il \textbf{supporto}
di una funzione $f$ come:

$$
    \sup f = \{ x \in \mathbb{R} : f(x) \neq 0 \}.
$$

%% TODO: qui ci stanno alcune discrepanze tra Vinti e Fagiolo, va ricontrollato.
\begin{theorem}
    Sia $f \in L^2(\mathbb{R}) \cap C(\mathbb{R})$ una funzione continua, ad energia finita e
    a banda limitata, i.e., tale che
    $\sup \hat{f} \subset \left[ -\pi W, \ \pi W\right], W > 0$.
    Allora vale la seguente formula di interpolazione.

    $$
        f(t) = \sum_{k = - \infty}^{+\infty} f \left(\frac{k}{W}\right) sinc \left[ \pi(Wt - k) \right], \text{ con } t \in \mathbb{R}
    $$

    dove:

    $$
        sinc(t) :=
        \left\{ \begin{array}{cl}
            \frac{sin(\pi t)}{\pi t} & \ t \neq 0 \\
            1                        & \ t = 0
        \end{array} \right.
    $$
\end{theorem}


\begin{figure}[H]
    \centering
    \begin{tikzpicture}[x=0.75pt,y=0.75pt,yscale=-1,xscale=1]
        %uncomment if require: \path (0,300); %set diagram left start at 0, and has height of 300

        %Shape: Axis 2D [id:dp6614397845308091] 
        \draw  (204,183.5) -- (379,183.5)(291,82) -- (291,196.5) (372,178.5) -- (379,183.5) -- (372,188.5) (286,89) -- (291,82) -- (296,89)  ;
        %Straight Lines [id:da7696580538613964] 
        \draw    (250,161.25) -- (250,183.25) ;
        %Straight Lines [id:da059819649685002974] 
        \draw    (331,161.25) -- (331,183.25) ;
        %Straight Lines [id:da32947787237217474] 
        \draw    (331,161.25) -- (250,161.25) ;

        % Text Node
        \draw (222.5,191.5) node [anchor=north west][inner sep=0.75pt]   [align=left] {$\displaystyle -\pi W$};
        % Text Node
        \draw (316,192) node [anchor=north west][inner sep=0.75pt]   [align=left] {$\displaystyle \pi W$};
    \end{tikzpicture}
    \caption{Una funzione a banda limitata}
\end{figure}

\begin{figure}[H]
    \centering
    \begin{tikzpicture}[x=0.75pt,y=0.75pt,yscale=-1,xscale=1]
        %uncomment if require: \path (0,300); %set diagram left start at 0, and has height of 300

        %Shape: Axis 2D [id:dp6614397845308091] 
        \draw  (204,230.75) -- (542.5,230.75)(306,82) -- (306,250.25) (535.5,225.75) -- (542.5,230.75) -- (535.5,235.75) (301,89) -- (306,82) -- (311,89)  ;
        %Straight Lines [id:da059819649685002974] 
        \draw    (421,226.75) -- (421,237.25) ;
        %Straight Lines [id:da333995992515034] 
        \draw    (441,226.75) -- (441,237.25) ;
        %Straight Lines [id:da4908605908983048] 
        \draw    (461,226.75) -- (461,237.25) ;
        %Straight Lines [id:da5180908391730039] 
        \draw    (400,226.75) -- (400,237.25) ;
        %Curve Lines [id:da6657246696267412] 
        \draw    (254,118.5) .. controls (269.59,106.81) and (338.93,97.87) .. (337.5,120.75) .. controls (336.07,143.63) and (365.6,210.17) .. (385,212.25) .. controls (404.4,214.33) and (425,145.75) .. (443,135.25) .. controls (461,124.75) and (493.73,190.83) .. (506.5,181.25) ;
        %Shape: Circle [id:dp878091919683829] 
        \draw  [color={rgb, 255:red, 208; green, 2; blue, 27 }  ,draw opacity=1 ][fill={rgb, 255:red, 208; green, 2; blue, 27 }  ,fill opacity=1 ] (397,202) .. controls (397,200.62) and (398.12,199.5) .. (399.5,199.5) .. controls (400.88,199.5) and (402,200.62) .. (402,202) .. controls (402,203.38) and (400.88,204.5) .. (399.5,204.5) .. controls (398.12,204.5) and (397,203.38) .. (397,202) -- cycle ;
        %Shape: Circle [id:dp017384754588354534] 
        \draw  [color={rgb, 255:red, 208; green, 2; blue, 27 }  ,draw opacity=1 ][fill={rgb, 255:red, 208; green, 2; blue, 27 }  ,fill opacity=1 ] (417,169.5) .. controls (417,168.12) and (418.12,167) .. (419.5,167) .. controls (420.88,167) and (422,168.12) .. (422,169.5) .. controls (422,170.88) and (420.88,172) .. (419.5,172) .. controls (418.12,172) and (417,170.88) .. (417,169.5) -- cycle ;
        %Shape: Circle [id:dp3580711979696489] 
        \draw  [color={rgb, 255:red, 208; green, 2; blue, 27 }  ,draw opacity=1 ][fill={rgb, 255:red, 208; green, 2; blue, 27 }  ,fill opacity=1 ] (436,139.5) .. controls (436,138.12) and (437.12,137) .. (438.5,137) .. controls (439.88,137) and (441,138.12) .. (441,139.5) .. controls (441,140.88) and (439.88,142) .. (438.5,142) .. controls (437.12,142) and (436,140.88) .. (436,139.5) -- cycle ;
        %Shape: Circle [id:dp2482434926915713] 
        \draw  [color={rgb, 255:red, 208; green, 2; blue, 27 }  ,draw opacity=1 ][fill={rgb, 255:red, 208; green, 2; blue, 27 }  ,fill opacity=1 ] (458,141.5) .. controls (458,140.12) and (459.12,139) .. (460.5,139) .. controls (461.88,139) and (463,140.12) .. (463,141.5) .. controls (463,142.88) and (461.88,144) .. (460.5,144) .. controls (459.12,144) and (458,142.88) .. (458,141.5) -- cycle ;

        % Text Node
        \draw (412.5,243) node [anchor=north west][inner sep=0.75pt]  [font=\scriptsize] [align=left] {$\displaystyle \frac{k}{W}$};
        % Text Node
        \draw (435,243) node [anchor=north west][inner sep=0.75pt]  [font=\scriptsize] [align=left] {$\displaystyle \frac{k}{W}$};
    \end{tikzpicture}
    \caption{Esempio di applicazione del campionamento}
\end{figure}

In altre parole, il teorema del campionamento ci dice che, da una famiglia discreta di
valori campione, assunti in modo uniformemente spaziato su tutto l’asse reale, è possibile
ricostruire in modo esatto, un segnale ad energia finita (ovvero in $L^2(\mathbb{R})$) e a
banda limitata, mediante la serie interpolante introdotta sopra.\\

$\Delta t = 1/W$ denota l’\textit{intervallo temporale} tra i campioni, $\pi W/ 2\pi = w/2$ denota
l’\textit{ampiezza di banda} di $f$, che e misurata in cicli per unita di tempo (Hertz) e
$R = 1 /\Delta t$ denota l’\textit{ordine di campionamento} (\textit{sampling rate}),
misurato in campioni per unita di tempo.

H. Nyquist mise in rilievo il significato dell’“intervallo”(numero) $1/W$ nella telegrafia
e Shannon lo chiamo \textit{intervallo di Nyquist} corrispondente alla frequenza di
banda $[−\pi W, \pi W]$. L’ordine di campionamento $R$, legato all’intervallo di Nyquist,
rappresenta un minimo teorico per la ricostruzione del segnale.
Per essere più precisi, il teorema sampling costituisce un algoritmo che permette di
rispondere alle due seguenti domande:

\begin{enumerate}
    \item \textit{Quanti campioni sono necessari per assicurare che l’informazione
              contenuta nel segnale venga preservata, durante il processo di ricostruzione?}\\

          Se il segnale contiene componenti di alte frequenze, è necessario campionarlo almeno
          due volte la massima frequenza del segnale per evitare perdite di informazione,
          cioe $f_s \ge 2 f_H$ , dove $f_s$ e la frequenza sampling, $f_H$ e la piu alta
          frequenza del segnale e $f_s = 2 f_H$ è la frequenza (ordine) di Nyquist.
          Campionare ad un ordine più alto di $2 f_H$ significa utilizzare una successione
          più fine di valori campione, e questo è il caso del cosiddetto sovracampionamento (oversampling).
          Nella pratica il sovracampionamento deve essere effettuato molto spesso, poichè
          un fattore correttivo deve essere introdotto per compensare il fatto che il
          campionamento e l’interpolazione non possono coincidere con i corrispondenti valori
          teorici; questo è il caso del cosiddetto \textit{round-off error} o \textit{errore di
              quantizzazione}. Inoltre un altro errore temporale si presenta quando i campioni
          non possono essere presi esattamente agli istanti temporali $k/W$; questo errore è
          chiamato \textit{time-jitter error}.
          Un esempio concreto di sovracampionamento è rappresentato dal un lettore di compact disk.
          Infatti considerando che la piu alta frequenza
          udibile è approssimativamente intorno a 18.000 cicli al secondo (Hertz),
          a seconda dell’ascoltatore, in base alla teoria precedente un segnale audio
          deve essere campionato almeno 36.000 volte al secondo per far sì che
          il segnale venga ricostruito completamente. L’ordine di campionamento
          dei primi lettori di compact disk era di circa 44.000 campioni al secondo.
          Altre volte capita di non avere a disposizione un numero ragionevole di valori
          campione per ricostruire il segnale completamente. Questo
          avviene quando non conosciamo il segnale originale, ma dobbiamo ricostruirlo dalle
          informazioni che abbiamo a disposizione del segnale (cioè
          dai suoi valori campione). In questo caso, cioè quando il numero di
          informazioni (valori campione) è scarso, la distanza tra i nodi è più
          grande dell’intervallo di Nyquist ed abbiamo il cosiddetto sottocampionamento
          (undersampling). In questo caso si presenta il fenomeno di
          aliasing e sostanzialmente cio che succede è che le replicanti spettrali si
          sovrappongono.

          \begin{figure}[H]
              \centering
              \includegraphics[width=8cm, keepaspectratio]{capitoli/immagini/imgs/aliasing_tajmahal.jpg}
              \caption{Nella figura possiamo apprezzare un esempio di aliasing.}
          \end{figure}

    \item \textit{Dato un segnale, campionato ad un certo fissato ordine di campionamento,
              quali frequenze non devono essere contenute nel segnale al fine di preservare
              l’informazione?}\\

          Questa è la situazione inversa rispetto a quella presentata nella prima
          domanda. La risposta è che un segnale campionato ad un fissato ordine
          di campionamento non può contenere componenti a frequenze superiori
          alla metà dell’ordine di campionamento, cioe: $f_H \leq f_s/2$, dove
          $f_H$ è la frequenza più alta contenuta nel segnale e $f_s$ è la frequenza
          sampling. In altre parole, la più alta frequenza che puo essere accuratamente
          rappresentata è la metà della frequenza sampling. Quindi il teorema
          sampling fornisce un limite superiore per la più alta frequenza $f_H$.
          Spieghiamolo con un esempio: supponiamo di utilizzare un CD-rom con un
          fissato ordine di campionamento di 44.1KHz. Nyquist afferma che la più
          alta frequenza che può essere riprodotta da questo campionamento è di
          22.05KHz. Infatti le frequenze superiori a 22.05KHz producono aliasing.
          Per evitare questo, può essere utilizzato un filtro passa basso per bloccare
          le frequenze superiori a 22.05KHz. È importante notare che l’aliasing
          può essere prodotto sia da segnali campionati non correttamente (con
          un numero di campioni non sufficiente) che da frequenze troppo alte del
          segnale aventi un fissato ordine di campionamento.
\end{enumerate}

\subsection{Svantaggi}

Benchè il teorema del campionamento venga laramente impiegato in ambito applicativo
esso presenta svantaggi o limiti da questo punto di vista.

\begin{itemize}
    \item Le ipotesi fatte sul segnare da ricostruire $f$, ovvero che sia ad energia finita
          e a banda limitata, implicano che ci riduciamo a considerare una famiglia di segnali
          estremamente regolari e ci consente di ricostruire solo sengali appartenenti alla classe
          $C^{\infty}(\mathbb{R})$. Nelle applicazioni reali, i segnali difficilmente sono così
          regolari. Per esempio, nelle immagini, forti salti di livello nella scala di grigi
          (vedi bordi degli oggetti), matematicamente possono essere visti come delle
          discontinuità.
    \item Per ricostruire un segnale $f$ al tempo $t$ è necessario un numero di campioni infinito
          e il loro valore viene rilevato sia all'istante precedente a $t$ che a quelli successivi.
          Dal punto di vista applicativo, una prima difficoltà potrebbe essere quella di
          implementare una formula teorica che prevede infiniti termini.
          Il problema potrebbe essere risolto troncando la serie; tuttavia, questo introdurrebbe
          un errore (di troncamento) nella ricostruzione, che non garantirebbe piu
          l’interpolazione del segnale ma una mera approssimazione.
          Inoltre, nei problemi reali, i segnali sono noti solo nel “passato”, ovvero i
          valori campione di $f$ sono noti, in generale, solo negli istanti di tempo $k/W$
          precedenti a $t$. I valori di $f(k/W)$ nel “futuro”, ovvero quando $k/W > t$,
          non sono in generale disponibili.
    \item Per il principio di Hisemberg i segnali non possono essere contemporanemanete
          a durata e banda limitata, quindi i segnali nel mondo reale, in generale, non soddisfano
          le ipotesi del teorema del campionamento.
\end{itemize}

\subsection{Operatore Sampling Generalizzato}

Per poter rendere applicabile il teorema del campionamento ed evitare le
problematiche viste in precedenza, è stato studiato un \textit{operatore di tipo
    Sampling Generalizzato}, con lo scopo di fornire una versione approssimata del
teorema. L'idea di base è quella di sostituire la funzione $sinc$ con una
funzione $\varphi$ a supporto continuo e compatto, il che implica che la serie
si riduce ad una somma finita per ogni $t \in \mathbb{R}$ e dunque, per ricostruire
un dato segnale in un certo istante di tempo, sono necessari solo un numero
finito di valori campione.\\

Ora per dimostrare che i campioni possono essere assunti nel passato solo rispetto
a $t$ forniremo due definizioni:

\begin{definition}
    Sia $\varphi \in C_c(\mathbb{R})$ e $f \in C(\mathbb{R})$, definiamo la famiglia
    di operatori $(S_W^{\varphi}f)_{W>0}$ con

    $$
        (S_W^{\varphi}f)(t) := \sum_{k = - \infty}^{+\infty} f(\frac{k}{W}) \varphi(Wt-k), \ t \in \mathbb{R}
    $$
\end{definition}

L'operatore $S_W^{\varphi}$ è detto \textit{Operatore Sampling Generalizzato}, mentre
$\varphi$ è detta \textit{Nucleo} dell'operatore.

\begin{definition}
    Data la funzione $\varphi: \ \mathbb{R} \rightarrow \mathbb{R}$ e un numero
    $\beta > 0$, possiamo definire

    $$
        m_\beta(\varphi) := \sup_{u \in \mathbb{R}} \sum_{k \in \mathbb{Z}} |u-k|^\beta |\varphi(u-k)|
    $$

    che prende il nome di \textit{Momento assoluto discreto} di ordine $\beta$ della funzione
    $\varphi$.
\end{definition}

Grazie alla definizione appena introdotta otteniamo la seguente stima:

$$
    |(S_W^{\varphi}f)(t)| \leq \sum_{k = - \infty}^{+\infty} |f(\frac{k}{W})| |\varphi(Wt-k)| \leq ||f||_{\infty} \sum_{k = - \infty}^{+\infty} |\varphi(Wt-k)| \leq ||f||_{\infty} m_0(\varphi)
$$

$\forall t \in \mathbb{R}$.\\

Poichè $f$ è limitata e il momento $m_0(\varphi) \leq +\infty$ risulta che
l'operatore di sampling generalizzato risulta ben definito in $C(\mathbb{R})$.
Vale quindi il seguente risultato di approssimazione.

\begin{theorem}
    Sia $\varphi \in C_c(\mathbb{R})$ tale che

    \begin{equation}\label{eq:uno}
        \sum_{k = - \infty}^{+\infty} \varphi(u-k) = 1, \ \forall u \in \mathbb{R}
    \end{equation}

    Allora, se la funzione limitata $f: \mathbb{R} \rightarrow \mathbb{R}$ è continua
    in $t_0 \in \mathbb{R}$, risulta:

    $$
        \lim_{W \rightarrow \infty} (S_W^{\varphi}f)(t_0) = f(t_0)
    $$

    In particolare, se $f \in C(\mathbb{R})$ si ha:

    $$
        \lim_{W \rightarrow \infty} ||S_W^{\varphi}f - f||_\infty = 0
    $$
\end{theorem}

Da qui deriva il corollario che ci mostra com'è possibile effettuare un processo
di ricostruzione grazie a campioni rilevati esclusivamente nel passato.

\begin{corollary}
    Sia $\varphi \in C_c(\mathbb{R})$, con $\sup \varphi \subset \mathbb{R}^+$ e
    tale che sia soddisfatta la condizione (\ref{eq:uno}), allora per ogni funzione limitata
    $f: \mathbb{R} \rightarrow \mathbb{R}$, nei punti $t_0$ di continuità si ha:

    $$
        \lim_{W \rightarrow \infty} (S_W^{\varphi}f)(t_0) = \lim_{W \rightarrow \infty} \sum_{\frac{k}{W} < t_0} f(\frac{k}{W}) \varphi(Wt_0 - k) = f(t_0)
    $$
\end{corollary}

\begin{proof}
    Come conseguenza del fatto che $\varphi(Wt_0-k) = 0$ per $\frac{k}{W} \ge t_0$ risulta che

    $$
        (S_W^{\varphi}f)(t_0) = \sum_{\frac{k}{W} < t_0} f(\frac{k}{W}) \varphi(Wt_0 - k)
    $$

    Quindi la tesi segue banalmente per il teorema precedente.
\end{proof}

\paragraph{Note}
\begin{itemize}
    \item $\varphi$ si legge \textit{Phi}.
    \item $C_c(\mathbb{R})$ indica l'insieme delle funzioni continue a supporto
          compatto.
\end{itemize}